\subsection{Problema 2}


\begin{frame}[fragile]{Problema 2}
 Folositi plonoamele Taylor ca sa aratati faptul ca, pentru lungimile de unda mari, prin legea lui Plank se obtin rezultate aproximativ egale cu legea Rayleight-Jeans 
\end{frame}

\begin{frame}[fragile]{Problema 2}
 Putem folosi urmatoarea serie Taylor:
\begin{equation*}
    \large
    e^{x} = \sum_{n=0}^{\infty}i=\frac{x^n}{n!} =  1 + \frac{x}{1!} + \frac{x^2}{2!} + \frac{x^3}{3!} ... \:\: R = \infty
\end{equation*}

Avem:
\begin{equation*}
    f(\lambda) = \frac{8\pi h c \lambda^{-5}}{e^{hc / \lambda k T}-1}
\end{equation*}
Notam cu:
\begin{equation*}
x = hc / \lambda k T
\end{equation*}
\end{frame}

\begin{frame}[fragile]{Problema 2}

\begin{equation*}
\large
 f(\lambda) = \frac{8 \pi h c \lambda^{-5}}{[1+ \frac{h c}{\lambda k T}+ \frac{1}{2!}( \frac{h c}{\lambda k T})^2+ \frac{1}{3!}( \frac{h c}{\lambda k T})^3 + ...]-1}
\end{equation*}
\begin{equation*}
 \approx \frac{8 \pi h c \lambda^{-5}}{1+ \frac{hc}{\lambda K T} - 1}
\end{equation*}
\begin{equation*}
= \frac{8 \pi k T}{\lambda^{4}}
\end{equation*}
Aceasta este de fapt legea lui Rayleigh-Jeans.
Astfel, daca \(\lambda\) este mare si termenii seriei Taylor sunt mici in comparatie cu primul termen (nenul), aproximarea legii lui Planck folosind doar primul polinom Taylor conduce la legea lui Rayleigh-Jeans.
\end{frame}