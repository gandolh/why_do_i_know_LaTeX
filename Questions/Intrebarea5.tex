\subsection{Problema 5}


\begin{frame}[fragile]{Problema 5}
    \parindent 10mm 
    Observati cum graficul f se schimba pentru temperatura variabila (folosind legea lui Planck)
    \\
    In particular, faceti graficul f pentru stelele Betelgeuse (T = 3400 K), Procyon ( T = 6400 K), Sirius (T = 9200 K) si pentru Soare(rezolvat anterior).
    \\
    Cum variaza radiatia totala emisa (Aria de sub curba) in functie de T.
    Folositi graficele ca sa argumentati de ce Sirius se numeste steaua albastra si Betelgeuse steaua rosie.
    
\end{frame}

\begin{frame}[fragile]{Problema 5}
    Folosind legea lui Plank putem observa cum se schimba graficul functiei f in functie de temeratura stelelor specificate in enunt.
    \centering
    
     \includegraphics[width=8cm, height=5cm]{graphics/Graficex52.jpeg}
    
\end{frame}

\begin{frame}[fragile]{Problema 5}
    
    Cu cat creste temperatura, cu atat aria de sub curba creste, cum era de asteptat, cu cat e mai fierbinte steaua cu atat emite mai multa energie\\
    
    Cu cat temperatura este mai mare cu atat valoarea lui \(\lambda\) scade(lungimea de unda este mai mica)\\
    
    Din acest motiv Sirius este o stea albastra iar Betelgeuse este o stea rosie, majoritatea luminii venita de la Sirius este de olungima de unda scurta.Asta se traduce intr-o frecventa mai mare ce se plaseaza in partea albastra a spectrului.In timp ce Betelgeuse are o frecventa mai joasa, situata in spectrul rosu.
    
    
\end{frame}