\subsection{Problema 1}


\begin{frame}[fragile]{Problema 1}
\parindent 10mm 
    Folositi regula lui L'Hopital pentru a demonstra ca legea lui Plank rezulta in urmatoarele:
    \begin{equation}
        \lim_{\lambda\to 0^+} f(\lambda) = 0 \:\:si  \lim_{\lambda\to \infty} f(\lambda) = 0
    \end{equation}
    
    si astfel demonstrand ca legea lui Plank este mai buna decat Legea Rayleigh-Jeans pentru lungimile de unda scurte.
\end{frame}

\begin{frame}[fragile]{Problema 1}
\parindent 10mm 
\Large
    \begin{equation*}
    
        f(\lambda) = \frac{8\pi h c \lambda^{-5}}{e^{h c/\lambda k T}-1}=
        \frac{a\lambda^{-5}}{e^{b/(\lambda T)}-1}
        \\
        \vspace{1ex}%
        \small
        Notam cu: \\
        a = 8 \pi h c \\
        \vspace{1ex}%
        b = \dfrac{hc}{k}
        
    \end{equation*}
\end{frame}

\begin{frame}[fragile]{Problema 1}

\centering
    \begin{equation*}
    \lim_{\lambda\to 0^+} f(\lambda) = \lim_{\lambda\to 0^+}  \frac{a\lambda^{-5}}{e^{b/(\lambda T)}-1}\rightarrow\frac{\infty}{\infty} 
   \end{equation*}
    si :
    \begin{equation*}
    \lim_{\lambda\to \infty} f(\lambda) = \lim_{\lambda\to \infty} \frac{a \lambda^{-5}}{e^{b/(\lambda T)}-1}\rightarrow \frac{0}{0}
        
    \end{equation*}
    In ambele cazuri se poate folosi regula lui L'hopital
\end{frame}

\begin{frame}[fragile]{Problema 1}
\centering
    Vom rezolva limita catre infinit:
    \begin{equation*}
    \lim_{\lambda\to \infty} f(\lambda) = \lim_{\lambda\to \infty} \frac{\frac{d}{dx}(a \lambda^{-5})}{\frac{d}{dx}e^{b/(\lambda T)}-1}
    \end{equation*}
    \begin{equation*}
    = \lim_{\lambda\to \infty} \frac{-5 a \lambda^{-6}}{e^{b/\lambda T}[-\frac{b}{(\lambda T)^2}T]}
    \end{equation*}
    \begin{equation*}
     = \frac{-5 a T^2}{-b T}\lim_{\lambda\to \infty}\frac{\lambda^2 \lambda^{-6}}{e^{b/\lambda T}} 
     \end{equation*}
    \begin{equation*}
    = \frac{5 a T}{b} \lim_{\lambda\to \infty} \frac{\lambda^{-4}}{e^{b/{\lambda T}}} = \frac{0}{1} = 0
    \end{equation*}
\end{frame}

\begin{frame}[fragile]{Problema 1}
\centering
    Vom rezolva limita catre \(0^+\):
    \begin{equation*}
    \lim_{\lambda\to 0^+} f(\lambda) = \frac{5 a T}{b} \lim_{\lambda\to 0^+} \frac{\lambda^{-4}}{e^{b/(\lambda T)}} \rightarrow \frac{\infty}{\infty}
    \end{equation*}
    \begin{equation*}
    \lim_{\lambda\to 0^+} f(\lambda) =  \frac{5 a T}{b} \lim_{\lambda\to 0^+} \frac{-4 \lambda^{-5}}{[-\frac{b}{(\lambda T)^2}T]}
    \end{equation*}
    \begin{equation*}
    =\frac{20 a T^{2}}{b^2} \lim_{\lambda\to 0^+} \frac{\lambda^{-3}}{e^{b/\lambda T}} \rightarrow \frac{\infty}{\infty}
    \end{equation*}
    Notam cu:
    \begin{equation*}
    k_1 = \frac{20 a T^{2}}{b^2}
    \end{equation*}
\end{frame}


\begin{frame}[fragile]{Problema 1}
\centering
  Se va folosi L'hopital de mai multe ori iar constantele rezultate din aplicarea regulii se vor nota cu \( k_i = 1,2, ... ,n \) si se vor scoate in fata limitei \\
    \begin{equation*}
    \lim_{\lambda\to 0^+} f(\lambda) = k_1 \lim_{\lambda\to 0^+} \frac{\lambda^{-3}}{e^{b/\lambda T}}
    \end{equation*}
    \begin{equation*}
    = k_2 \lim_{\lambda\to 0^+} \frac{\lambda^{-2}}{e^{b/\lambda T}}
    \end{equation*}
    \begin{equation*}
    = k_3 \lim_{\lambda\to 0^+} \frac{\lambda^{-1}}{e^{b/\lambda T}}
    \end{equation*}
    \begin{equation*}
    = k_4 \lim_{\lambda\to 0^+} \frac{1}{e^{b/\lambda T}} = 0
    \end{equation*}
\end{frame}
