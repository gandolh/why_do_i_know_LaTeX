

\subsection{Problema 3}


\begin{frame}[fragile]{Problema 3}
\parindent 10mm 
    Ilustrati graficul f pentru ambele legi in paralel, si comentati similaritatile si diferentele.
    \\
    Folositi T= 5700K (temperatura soarelui)
    \\
    Puteti sa schimbati din metri in micrometri: \(1\mu m = 10^{-6} m \)
    
\end{frame}

\begin{frame}[fragile]{Problema 3}
    Pentru a converti \(\mu\) , o sa inmultim \lambda (m) * 10^{6}(\(\mu\)m/m)=\lambda(\mu m)
    
\end{frame}


\begin{frame}[fragile]{Problema 3}
\centering
\begin{minipage}{0.45\textwidth}
\centering
 \includegraphics[width=5cm, height=3cm]{graphics/poza_grafic1.jpeg}
\end{minipage}%
\hfill
\begin{minipage}{0.45\textwidth}
\centering
\begin{tabular}{|p{\textwidth}}
\includegraphics[width=5cm, height=3cm]{graphics/poa_grafic2.jpeg}
\end{tabular}
\end{minipage}%

    
\end{frame}

\begin{frame}[fragile]{Problema 3}
    \begin{enumerate}
    \item Prima figura arata faptul ca cele doua legi sunt similare pentru \(\lambda\) de dimensiuni mari\\
    \item A doua figura arata faptul ca cele doua legi sunt foarte diferite pentru lungimi de unda scurte\\
    \item Legea lui Plank are punctul de maxim la \(\lambda\)=0.5\(\mu\)m in timp ce legea Rayleigh-Jeans nu are maxim sau minim
    \end{enumerate}
\end{frame}