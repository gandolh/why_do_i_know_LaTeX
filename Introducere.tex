\section{Introducere}

\begin{frame}[fragile]{Introducere: Radiatia stelelor}
%You can change fragile by standout
\parindent 10mm 
Orice obiect emite radiatii la incalzire, dar un "blackbody" reprezinta un sistem care absoarbe toata radiatia cu care intra in contact.
\\
Un bun exemplu de blackbody ar fi o suprafata negru mat sau o cavitate mare ce contine o gaura mica in peretii acesteia.
\\
    In esenta un "blackbody" absoarbe toata lumina primita si emite o lumina independenta. 
\\
    Lumina ce este transmisa de obiect este doar dependenta de propia temperatura, o buna aproximatie  este o gaura mica printr-un obiect gol.

\end{frame}


\begin{frame}[fragile]{Introducere: Aproximarea blackbody}
%You can change fragile by standout

\begin{minipage}{0.45\textwidth}
    \begin{enumerate}
    \item Cum am specificat anterior, o buna aproximarea a unui blackbody este un obiect gol cu o gaura mica.
    \item Gaura absoarbe perfect radiatiile.
    \item Natura radiatiilor ce ies din cavitate depinde doar de temperatura cavitatii.
    \end{enumerate}
\end{minipage}%
\hfill
\begin{minipage}{0.45\textwidth}
\begin{tabular}{|p{\textwidth}}
\includegraphics[width=5cm, height=4cm]{blackbody_aprox.jpg}
\end{tabular}
\end{minipage}%

\end{frame}


\begin{frame}[fragile]{Introducere: De retinut}
 %You can change fragile by standout
\begin{center}
    \includegraphics[width=6cm, height=3cm]{graphics/rainbow_wavelength.png}
\vspace{1ex}%
\\
Temperatura soarelui este de 5700K, acest lucru situandu-l in spectrul vizibil de culoare.
\\
    Chiar daca radiatia acestuia este aproape de a fi o radiatie de tip blackbody, este in mare parte doar un rezultat al caldurii sale.
\\


\end{center}

\end{frame}

\begin{frame}[fragile]{Introducere: DADADA OTELU E VIATA MEA}
\begin{center}
\includegraphics[width=6cm, height=3cm]{graphics/star_wavelength.png}
\end{center}

Frecventa si lungimile de unda sunt corelate.
\\
\parindent 10mm 
Lungime de unda =  distanta periodica dintre unda si cicluri
\\
frecventa = unde sau cicluri pe secunda.
\\
\noindent 
Pentru lumina sau radiatii, cu cat e mai scurta lungimea de unda cu atat e mai mare frecventa (si energia).
\\
Ca in urmatoarea expresie: 
\(f = \dfrac{c}{\lambda}\)

\end{frame}


\begin{frame}[fragile]{Introducere: CE FAC CU VIATA MEA?}

Propusa la finalul secolului 19, legea Rayleigh-Jeans exprima densitatea energiei unei unde \(\lambda\) de radiatie blackbody ca: 
\begin{equation}
    f(\lambda) = \frac{8 \pi k T}{\lambda^4}
\end{equation}
\begin{enumerate}
    \item Unde \(\lambda\) este masurata in metri
    \item Unde T este temperatura in Kelvin
    \item Unde k e constanta lui Boltzmann
\end{enumerate}

\end{frame}

\begin{frame}[fragile]{Introducere: PULA MEA}
\parindent 10mm 
Legea Rayleigh-Jeans este mult mai potrivita pentru masuratorile experimentale asupra lungimilor de unda mari, avand erori de calcul asupra lungimilor de unda mici.

Aceasta calcula \(f(\lambda) \rightarrow \liminf\) cand \(\lambda \rightarrow 0^+\) dar experimentele au demonstrat ca \(f(\lambda) \rightarrow 0\)

In 1900 Max Planck a formulat un model(Legea lui Plank) mai bun pentru radiatia blackbody:
\begin{equation}
f(\lambda) = \frac{8 \pi h c \lambda ^ -5}{e^{h c /(\lambda K T)}-1}   
\end{equation}
\begin{enumerate}
    \item Unde h e constanta lui Planck = \( 6.6262 * 10^{-34}\)
    \item Unde k este constanta lui Boltzmann = \( 1.3807 * 10^{-23}\)
    \item Unde c este viteza luminii = \(2.997925 * 10^8 \)
\end{enumerate}
\end{frame}